% Created 2018-01-15 Mon 09:40
\documentclass{tufte-handout}
\usepackage[utf8]{inputenc}
\usepackage[T1]{fontenc}
\usepackage{fixltx2e}
\usepackage{graphicx}
\usepackage{longtable}
\usepackage{float}
\usepackage{wrapfig}
\usepackage{rotating}
\usepackage[normalem]{ulem}
\usepackage{amsmath}
\usepackage{textcomp}
\usepackage{marvosym}
\usepackage{wasysym}
\usepackage{amssymb}
\usepackage{hyperref}
\tolerance=1000
\date{15 January 2018}
\title{PHYS 381: Homework 00}
\hypersetup{
  pdfkeywords={},
  pdfsubject={},
  pdfcreator={Emacs 25.1.1 (Org mode 8.2.10)}}
\begin{document}

\maketitle
\rule{\linewidth}{0.5pt}
\section{Getting access to the cluster}
\label{sec-1}
Point a web browser to \url{https://atlacamani.marietta.edu:8000} and log in. Your username is your MC email username (your email address without the @marietta.edu), and your initial password is your student ID.

\rule{\linewidth}{0.5pt}
\section{Change your initial password on the cluster}
\label{sec-2}
\begin{itemize}
\item Once you have logged in to the dashboard, select New -> Terminal.
\item In the terminal window, type `passwd`, and then follow the instructions to change your password.
\end{itemize}

\rule{\linewidth}{0.5pt}
\section{Getting access to a shell on your own computer}
\label{sec-3}
\begin{itemize}
\item If you have a Mac, find the Terminal app. By default, it's in the \emph{Other} group accessible in the Launchpad.
\item If you have a Windows machine, download and install Git for Windows. It includes a bash shell, and we will need it later anyway.
\end{itemize}
% Emacs 25.1.1 (Org mode 8.2.10)
\end{document}